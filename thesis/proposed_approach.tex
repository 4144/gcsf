\chapter{Proposed approach}

\section{Aim}

This project aims to improve the experience of using Google Drive specifically for power users. Regular users can benefit from it as well -- provided they follow some prerequisite steps in order to use the application.

If successful, it will drastically diminish the issues described in \ref{power_user_experience}.

\section{Summary}

In essence, Google Drive is nothing more than a remote storage system. All the operations that a user might want to execute on a local storage system (e.g. copying files, creating and organising directories, reading and writing data) have an equivalent on Google Drive.

Disk-based storage devices are organized by the operating system using a filesystem in order to keep track of all the data they contain. This happens behind the scences. Users can interact with the storage device using system calls. Some of the more popular ones are \codeword{read}, \codeword{write}, \codeword{close}, \codeword{wait}, \codeword{exec}, \codeword{fork}, \codeword{exit}, \codeword{kill}. Note that not all of these deal with file storage. Some of them are a proxy which expose different functionalities of the operating system.

An interesing concept comes to mind: why not model a Google Drive account in such a way that it behaves identically to a traditional filesystem? The only difference would be that instead of storing and reading data from a local disk, it would interact with Google's servers.
