\documentclass[a4paper]{article}

\usepackage{amsmath}
\usepackage{amsfonts}
\usepackage{amssymb}
\usepackage{graphicx}

\begin{document}
\pagenumbering{gobble}

\begin{center}
\Large\textbf{Abstract}
\end{center}

\hspace{10pt}

\normalsize

This thesis takes a look into the world of cloud storage services with a focus on Google Drive. Ultimately, it aims to provide a better way of interacting with this service.

\hspace{10pt}

Chapter 1 contains a brief introduction on the subject. Chapter 2 delves into the history and evolution of cloud storage systems throughout the past decade, carefully analysing their strengths and shortcomings.

\hspace{10pt}

Chapter 3 proposes an original alternative to the Google Drive web client. This alternative comes in the form of a virtual Unix file system named GCSF. Its purpose is to improve the experience of advanced users by supplying them with more control and a more cohesive integration with the operating system.

\hspace{10pt}

This claim is examined closely in Chapter 4, where GCSF is compared against existing tools. Their performance is benchmarked and their trade-offs scrutinized.

\hspace{10pt}

Finally, Chapter 5 concludes this thesis with a few closing thoughts and several areas of future improvement.

\hspace{10pt}

This work is the result of my own activity. I have neither given nor received unauthorized assistance on this work.

\hspace{10pt}

\parindent0pt
\textbf{Cluj-Napoca, 24 June 2018}
\hfill
\textbf{Pușcaș Sergiu Dan} \\
\hfill

\end{document}

